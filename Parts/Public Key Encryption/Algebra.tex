

\chapter{Algebra}
	
\section{Basics of Algebra}
	\subsection{Ring of integers $\mathbb{Z}$}
		A ring $(R,+,\cdot)$ is a set $R$ equipped with two binary operations $+$ and $\cdot$.
		A ring is satisfying the following three axioms (ring axioms):
		\begin{enumerate}
			\item $R$ is an \textbf{abelian group} under addition ($+$), meaning that for all $a,b,c \in R$:
			\begin{itemize}
				\item \textbf{Associative} under addition: $(a+b)+c = a+(b+c)$
				\item \textbf{Commutative} under addition: $a+b = b+a$
				\item Additive \textbf{identity}: There is an element $0 \in R$ such that $a+0 = a$
				\item Additive \textbf{inverse}: For each $a$ there exists an $-a \in R$ such that $a+(-a) = 0$
			\end{itemize}
			\item $R$ is an \textbf{monoid} under multiplication ($\cdot$), meaning that for all $a,b,c \in R$:
			\begin{itemize}
				\item \textbf{Associative} under multiplication: $(a \cdot b) \cdot c = a \cdot (b \cdot c)$
				\item Multiplicative \textbf{identity}: There is an element $1 \in R$ such that $a \cdot 1 = 1 \cdot a$
			\end{itemize}
			\item Multiplication is distributive with respect to addition, meaning that for all $a,b,c \in R$:
			\begin{itemize}
				\item \textbf{Left distributivity}: $a \cdot (b+c) = (a \cdot b) + (a \cdot c)$
				\item \textbf{Right distributivity}: $(b+c) \cdot a = (b \cdot a) + (c \cdot a)$
			\end{itemize}
		\end{enumerate}
		
		\subsubsection{Divisibility}
			Division operations are essential for the work in modern cryptography. We define $a$ \textit{divides} $b$, written $a|b$ as follow:
			\begin{center}
				$a|b$ $:\Leftrightarrow$ $\exists \lambda \in \mathbb{Z}$ so that $b = \lambda \cdot a$
			\end{center}
			If $a$ does not divide $b$, we write $a \nmid b$.
		
		\subsubsection{Euclidean division}
			

























