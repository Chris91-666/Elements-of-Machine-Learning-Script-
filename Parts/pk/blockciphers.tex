

\chapter{Block Ciphers}

\begin{definition}[Block Ciphers]\ \\
    Let $k,n$ be positive integers. A block cipher with key length $k$ and block length $n$ is an efficient keyed permutation 
    $$F: \{0,1\}^k \times \{0,1\}^n \to \{0,1\}^n$$
    
    \textbf{Remark:}
    \begin{itemize}
        \item The function $F_K \coloneq F(K,\cdot)$ is a bijection (its inverse is denoted $F_K^{-1}$).
        \item Both $F_K$ and $F_K^{-1}$ are efficiently computable given the key $K$.
        \item A secure block cipher should "behave as a pseudorandom permutation".\newline
    \end{itemize}
\end{definition}


\textbf{A Note on the Concrete Security Setting}\newline
The key length and block length of block ciphers are fixed $\to$ no varying "security parameter". In practice:
\begin{itemize}
    \item actual (not asymptotic) complexity of adversaries is considered.
    \item a block cipher is considered secure as long as no attack significantly faster than exhaustive key search exists.\newline
\end{itemize}


\begin{definition}[Security Definition - Indistinguishability from a Random Permutation]\ \\
    A block cipher $F$ with key length $k$ and block length $n$ is $(q,t,\epsilon)$-sprp secure if, for every probabilistic adversary $D$ that runs in time at most $t$ and deals at most $q$ oracle queries, one has
    $$Adv_F^{sprp}(D) \coloneq \vert Pr[D^{F_K(\cdot),F_K^{-1}(\cdot)}=1]-Pr[D^{f(\cdot),f^{-1}}=1] \vert \leq \epsilon,$$
    where the first probability is taken over the uniformly random draw of $K$ and the second one over the uniformly random draw of the permutation $f: \{0,1\}^n \to \{0,1\}^n$.\newline
    
    \textbf{Remark:}\newline
    $F$ is considered "secure" as long as $Adv_F^{sprp}(D) \leq c_1 \frac{t/T_F}{2^k} + c_2 \frac{q}{2^n}$, where $c_1$ and $c_2$ are small constants and $T_F$ is an upper bound on the time required to evaluate $F$.\newline
\end{definition}


\textbf{A word on Provable Security}
\begin{itemize}
    \item Typically, the security of block ciphers is not provable.
    \item The pseudorandomness of a block cipher is actually used as an assumption in security proofs for block cipher-based private-key algorithms.
    \item To build confidence in the pseudorandomness of a block cipher:
    \begin{itemize}
        \item decades of cryptanalysis;
        \item heuristical arguments of security against particular classes of attacks (generic attacks, differential or linear cryptanalysis, ...).\newline
    \end{itemize}
\end{itemize}

\newpage

\textbf{Design Principles}
\begin{itemize}
    \item Block ciphers should behave as pseudorandom permutations: a 1-bit change in the input should affect every bit of the output (avalanche effect)!
    \item But block ciphers should be efficient and have a short description.
    \item Confusion-diffusion paradigm: build a pseudorandom permutation from small random (or random-looking) permutations.
    \item In practice, the following process, called a round, is applied several times to the input block:
    \begin{itemize}
        \item divide the block in small chunks;
        \item apply small random-looking permutations (called S-boxes) to each chunk of data (confusion);
        \item mix the bits of the intermediate value to spread the local changes to the whole block (diffusion); this step can be a simple reordering of the bits or the application of a more complex (invertible) linear function.\newline
    \end{itemize}
\end{itemize}


\textbf{Substitution-Permutation Networks}
\begin{itemize}
    \item Practical instantiation of the confusion-diffusion paradigm.
    \item The S-boxes, key expansion algorithm and linear layer L are public.
    \item The key is expanded to several subkeys that are mixed with the intermediate values using a bitwise XOR.
    \item Subkeys are added before each round and after the last one.
    \item The construction is invertible since both the S-boxes and the linear mixing layer are invertible.\newline
\end{itemize}


\textbf{The Avalanche Effect}
\begin{itemize}
    \item Example of design criterion to get the avalanche effect:
    \begin{itemize}
        \item the S-boxes are chosen so that changing a single bit of the input changes at least two bits of the output.
        \item the mixing permutations (in that case simple bit reordering) are chosen so that the output bits of any given S-box are used as input of multiple S-boxes in the next round.
    \end{itemize}
    \item Consequence:
    \begin{itemize}
        \item a single bit difference between input blocks results in a difference of two bits after one round;
        \item the second condition ensures that the inputs of at least two S-boxes of the second round will differ by one bit;
        \item after the second round: at least 4 bits differ between the two blocks;
        \item in the best case, $2^r$ bits will be affected after $r$ rounds (actually less);
        \item this gives a lower bound for the number of rounds of an SPN: $$r \geq \lceil \log_2(n) \rceil$$\newline
    \end{itemize}
\end{itemize}


\textbf{Feistel Networks}
\begin{itemize}
    \item Generic iterated structure to build a PRP from round functions.
    \item Typically, round functions are also constructed from (possibly non-invertible) S-boxes and linear mixing layers.
    \item Description of a round:
    \begin{itemize}
        \item break the block in two equal halves $L_i$ and $R_i$;
        \item output $(L_{i+1},R_{i+1})$ defined as $L_{i+1} = R_i$ and $R_{i+1} = L_i \oplus f_i(R_i)$, where $f_i$ denotes the $i$-th (public) round function instantiated with a (secret) key $K$.
    \end{itemize}
    \item Inherently invertible: $R_i = L_{i+1}$ and $L_i = R_{i+1} \oplus f_i(L_{i+1})$.\newline
\end{itemize}


\begin{theorem}[Feistel Networks]\ \\
Assuming the round functions are uniformly random and independent
functions from $\{0,1\}^n$ to $\{0,1\}^n$, then:
\begin{itemize}
    \item the $3$-round Feistel construction is $(q,\infty,\frac{q^2}{2^n})$-prp secure;
    \item the $4$-round Feistel construction is $(q,\infty,\frac{q^2}{2^n})$-sprp secure;
    \item the $6$-round Feistel construction is $(q,\infty,\frac{9q}{2^n})$-sprp secure secure as long as $q \leq 2^{n-7}$.\newline
\end{itemize}

\textbf{Interpretation of the theorem:}
\begin{itemize}
    \item actual block ciphers have simple round functions that are not random or pseudorandom;
    \item however, the result justifies the soundness of the Feistel structure;
    \item it provides lower bounds for the number of rounds that have to be used by an actual block cipher.\newline
\end{itemize}
\end{theorem}


\textbf{Building Blocks of a Block Cipher}\ \\
To summarize, the specification of a block cipher generally contains the following ingredients:
\begin{itemize}
    \item a high-level (iterated) structure (SPN, Feistel scheme,. . . );
    \item a key-schedule algorithm to derive sub-keys from a master key;
    \item small S-boxes that must be non-linear;
    \item an efficient linear layer to properly spread local changes from the application of the S-boxes.\newline
\end{itemize}

