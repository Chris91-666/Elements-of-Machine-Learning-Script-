

\chapter{Principles}

	\section{Lessons from Historic Ciphers}
		\begin{itemize}
			\item Ciphers without keys or small key-space cannot recover from compromise
			\item Ciphertext should not reveal statistical information about plaintexts
			\item Ciphertexts should not reveal any (efficiently computable) information about the plaintext
			\item Further thoughts: Many historic ciphers immediately broken given a single message-ciphertext pair
			\item Ciphers were designed with limited imagination
			\item \textbf{How do you defend against an adversary who is smarter than you?}
		\end{itemize}
	
	\section{Kerkhoff's Principle}
		\begin{center}
			\textbf{The cipher method must not be required to be secret, and it must be able to fall into the hands of the enemy without inconvenience.}
		\end{center}
		\begin{itemize}
			\item Designs which violate Kerkhoff's principle hope to achieve Security by Obscurity
			\item Today this is considered bad cryptographic design
			\item It is easier to switch a compromised key than a compromised design
		\end{itemize}
	
	\section{Principles of Modern Cryptography}
		\begin{itemize}
			\item \textbf{Formal Definitions/Models}
				\begin{itemize}
					\item What constitutes a successful attack?
					\item What is the goal of an attack?
				\end{itemize}
			\item \textbf{Precise Assumptions}
				\begin{itemize}
					\item Underlying hard problem?
				\end{itemize}
			\item \textbf{Formal Proofs}
				\begin{itemize}
					\item Show that a successful attack actually violates the assumptions
				\end{itemize}
		\end{itemize}
	
	\section{Formal Models}
		\begin{itemize}
			\item Example: Encryption
			\item We will generally define security by defining what it means that a scheme is insecure.
			\begin{itemize}
				\item What is the computational power of the adversary?
				\item What information is given to the adversary?
				\begin{itemize}
					\item Only Ciphertext
					\item A message and a ciphertext encrypting this message
					\item A ciphertext encrypting a message chosen by the adversary?
					\item Access to a “decryption oracle”
				\end{itemize}
				\item What is the goal of the adversary?
				\begin{itemize}
					\item Recover the key
					\item Recover the message
					\item Recover parts of the message
					\item Distinguish encryptions of two messages
				\end{itemize}
			\end{itemize}
		\end{itemize}
	
	\section{Precise Assumptions}
		\begin{itemize}
			\item Should it be impossible to break the cipher only hard?
			\item How can we mathematically define hard problems?
			\begin{itemize}
				\item Complexity Theory!
				\item Example: Factor large numbers
			\end{itemize}
			\item A good assumption should be simple to state but hard to break
		\end{itemize}
		\textbf{Precise Assumptions let us}
		\begin{itemize}
			\item Falsify an assumption by cryptanalysis
			\item Build confidence by lack of cryptanalysis
			\item Compare schemes by their underlying assumptions
			\item Understand the necessity of the assumption
		\end{itemize}
	
	\section{Summary}
		\begin{itemize}
			\item Kerkhoff’s principle: Only the key should be secret, not the scheme itself
			\item Modern cryptography evolved into a discipline following scientific principles
			\item It lets us design schemes secure against adversaries smarter than ourselves!
		\end{itemize}

















